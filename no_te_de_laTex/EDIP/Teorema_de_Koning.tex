%Plantilla para realizar los ejercicios de manera individual y que sean visibles y útiles también de manera modular
\ifdefined\COMPLETE

%%%%%%%%%%%%%%%%%%%%% aquí escrbís lo que queráis que tenga el documento %%%%%%%
El teorema de Köning dice que $\sum_{i=1}^{k} f_i \big( x_i - a \big) =\sum_{i=1}^{k} f_i \big( x_i - \bar{x} \big) + \big( a - \bar{x}\big)^2  $
\subsection*{Demostración }
$$\sum_{i=1}^{k} f_i \big( x_i - a \big) = \sum_{i=1}^{k}f_i\big( x_i - a )^2 -  \sum_{i=1}^{k}f_i 2ax_i + \sum_{i=1}^{k}f_i a^2 $$
Aplicando la definición de media $\bar{x}=  \sum_{i=1}^{k}f_ix_i $ y sacando factor común $a^2$ en el último sumando y sabiendo que $ \sum_{i=1}^{k}f_i = 1 $ resulta:

$$ \sum_{i=1}^{k}f_ix_{i}^2 - 2a\bar{x} + a^2 $$
 sumando $ \bar{x}^2- \bar{x}^2$ y asociando en forma de binomio, queda:

 $$ \sum_{i=1}^{k}f_ix_{i}^2  -  \bar{x} + \big( a - \bar{x} \big)^2 $$
 volvemos a sumar $ \bar{x}^2- \bar{x}^2$ y aplicando la definición de media y la de sumatoria de frecuencias relativas resulta:
\par
$$ \sum_{i=1}^{k}f_ix_{i}^2  - 2\bar{x}\sum_{i=1}^{k}f_ix_{i}^2 + \sum_{i=1}^{k}f_i\bar{x}^2 + \big( a - \bar{x} \big)^2 $$
Sacando factor común las sumatorias y viendo que eso es el desarrollo de una un binomio:
$$\sum_{i=1}^{k} f_i \big( x_i - \bar{x} \big) + \big( a - \bar{x}\big)^2  $$
Que es lo que pretendíamos demostrar. 


%%%%%%%%%%%% fin de vuestro documento %%%%%%%%%%%%%%%%%%%%%%%%%%%%%%%%
\else 
\def\COMPLETE{}

\documentclass[a4paper , 11pt, spanish ]{article}
% Codificación e idioma, para las tildes crucial
\usepackage{paquetes}
\title {Teorema de Koning }
\author{Blanca}


\begin{document}
\maketitle 
%Plantilla para realizar los ejercicios de manera individual y que sean visibles y útiles también de manera modular
\ifdefined\COMPLETE

%%%%%%%%%%%%%%%%%%%%% aquí escrbís lo que queráis que tenga el documento %%%%%%%
El teorema de Köning dice que $\sum_{i=1}^{k} f_i \big( x_i - a \big) =\sum_{i=1}^{k} f_i \big( x_i - \bar{x} \big) + \big( a - \bar{x}\big)^2  $
\subsection*{Demostración }
$$\sum_{i=1}^{k} f_i \big( x_i - a \big) = \sum_{i=1}^{k}f_i\big( x_i - a )^2 -  \sum_{i=1}^{k}f_i 2ax_i + \sum_{i=1}^{k}f_i a^2 $$
Aplicando la definición de media $\bar{x}=  \sum_{i=1}^{k}f_ix_i $ y sacando factor común $a^2$ en el último sumando y sabiendo que $ \sum_{i=1}^{k}f_i = 1 $ resulta:

$$ \sum_{i=1}^{k}f_ix_{i}^2 - 2a\bar{x} + a^2 $$
 sumando $ \bar{x}^2- \bar{x}^2$ y asociando en forma de binomio, queda:

 $$ \sum_{i=1}^{k}f_ix_{i}^2  -  \bar{x} + \big( a - \bar{x} \big)^2 $$
 volvemos a sumar $ \bar{x}^2- \bar{x}^2$ y aplicando la definición de media y la de sumatoria de frecuencias relativas resulta:
\par
$$ \sum_{i=1}^{k}f_ix_{i}^2  - 2\bar{x}\sum_{i=1}^{k}f_ix_{i}^2 + \sum_{i=1}^{k}f_i\bar{x}^2 + \big( a - \bar{x} \big)^2 $$
Sacando factor común las sumatorias y viendo que eso es el desarrollo de una un binomio:
$$\sum_{i=1}^{k} f_i \big( x_i - \bar{x} \big) + \big( a - \bar{x}\big)^2  $$
Que es lo que pretendíamos demostrar. 


%%%%%%%%%%%% fin de vuestro documento %%%%%%%%%%%%%%%%%%%%%%%%%%%%%%%%
\else 
\def\COMPLETE{}

\documentclass[a4paper , 11pt, spanish ]{article}
% Codificación e idioma, para las tildes crucial
\usepackage{paquetes}
\title {Teorema de Koning }
\author{Blanca}


\begin{document}
\maketitle 
%Plantilla para realizar los ejercicios de manera individual y que sean visibles y útiles también de manera modular
\ifdefined\COMPLETE

%%%%%%%%%%%%%%%%%%%%% aquí escrbís lo que queráis que tenga el documento %%%%%%%
El teorema de Köning dice que $\sum_{i=1}^{k} f_i \big( x_i - a \big) =\sum_{i=1}^{k} f_i \big( x_i - \bar{x} \big) + \big( a - \bar{x}\big)^2  $
\subsection*{Demostración }
$$\sum_{i=1}^{k} f_i \big( x_i - a \big) = \sum_{i=1}^{k}f_i\big( x_i - a )^2 -  \sum_{i=1}^{k}f_i 2ax_i + \sum_{i=1}^{k}f_i a^2 $$
Aplicando la definición de media $\bar{x}=  \sum_{i=1}^{k}f_ix_i $ y sacando factor común $a^2$ en el último sumando y sabiendo que $ \sum_{i=1}^{k}f_i = 1 $ resulta:

$$ \sum_{i=1}^{k}f_ix_{i}^2 - 2a\bar{x} + a^2 $$
 sumando $ \bar{x}^2- \bar{x}^2$ y asociando en forma de binomio, queda:

 $$ \sum_{i=1}^{k}f_ix_{i}^2  -  \bar{x} + \big( a - \bar{x} \big)^2 $$
 volvemos a sumar $ \bar{x}^2- \bar{x}^2$ y aplicando la definición de media y la de sumatoria de frecuencias relativas resulta:
\par
$$ \sum_{i=1}^{k}f_ix_{i}^2  - 2\bar{x}\sum_{i=1}^{k}f_ix_{i}^2 + \sum_{i=1}^{k}f_i\bar{x}^2 + \big( a - \bar{x} \big)^2 $$
Sacando factor común las sumatorias y viendo que eso es el desarrollo de una un binomio:
$$\sum_{i=1}^{k} f_i \big( x_i - \bar{x} \big) + \big( a - \bar{x}\big)^2  $$
Que es lo que pretendíamos demostrar. 


%%%%%%%%%%%% fin de vuestro documento %%%%%%%%%%%%%%%%%%%%%%%%%%%%%%%%
\else 
\def\COMPLETE{}

\documentclass[a4paper , 11pt, spanish ]{article}
% Codificación e idioma, para las tildes crucial
\usepackage{paquetes}
\title {Teorema de Koning }
\author{Blanca}


\begin{document}
\maketitle 
%Plantilla para realizar los ejercicios de manera individual y que sean visibles y útiles también de manera modular
\ifdefined\COMPLETE

%%%%%%%%%%%%%%%%%%%%% aquí escrbís lo que queráis que tenga el documento %%%%%%%
El teorema de Köning dice que $\sum_{i=1}^{k} f_i \big( x_i - a \big) =\sum_{i=1}^{k} f_i \big( x_i - \bar{x} \big) + \big( a - \bar{x}\big)^2  $
\subsection*{Demostración }
$$\sum_{i=1}^{k} f_i \big( x_i - a \big) = \sum_{i=1}^{k}f_i\big( x_i - a )^2 -  \sum_{i=1}^{k}f_i 2ax_i + \sum_{i=1}^{k}f_i a^2 $$
Aplicando la definición de media $\bar{x}=  \sum_{i=1}^{k}f_ix_i $ y sacando factor común $a^2$ en el último sumando y sabiendo que $ \sum_{i=1}^{k}f_i = 1 $ resulta:

$$ \sum_{i=1}^{k}f_ix_{i}^2 - 2a\bar{x} + a^2 $$
 sumando $ \bar{x}^2- \bar{x}^2$ y asociando en forma de binomio, queda:

 $$ \sum_{i=1}^{k}f_ix_{i}^2  -  \bar{x} + \big( a - \bar{x} \big)^2 $$
 volvemos a sumar $ \bar{x}^2- \bar{x}^2$ y aplicando la definición de media y la de sumatoria de frecuencias relativas resulta:
\par
$$ \sum_{i=1}^{k}f_ix_{i}^2  - 2\bar{x}\sum_{i=1}^{k}f_ix_{i}^2 + \sum_{i=1}^{k}f_i\bar{x}^2 + \big( a - \bar{x} \big)^2 $$
Sacando factor común las sumatorias y viendo que eso es el desarrollo de una un binomio:
$$\sum_{i=1}^{k} f_i \big( x_i - \bar{x} \big) + \big( a - \bar{x}\big)^2  $$
Que es lo que pretendíamos demostrar. 


%%%%%%%%%%%% fin de vuestro documento %%%%%%%%%%%%%%%%%%%%%%%%%%%%%%%%
\else 
\def\COMPLETE{}

\documentclass[a4paper , 11pt, spanish ]{article}
% Codificación e idioma, para las tildes crucial
\usepackage{paquetes}
\title {Teorema de Koning }
\author{Blanca}


\begin{document}
\maketitle 
\input{./Teorema_de_Koning} %nombre de este documento 
\end{document}


\fi %nombre de este documento 
\end{document}


\fi %nombre de este documento 
\end{document}


\fi %nombre de este documento 
\end{document}


\fi